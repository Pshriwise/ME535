\documentclass[14pt]{beamer}
\mode<presentation>
{
\usepackage{color}
\usepackage{graphicx}
\usepackage{tikz}
\usetheme[white]{Wisconsin}
\setbeamercovered{transparent}
}

\begin{document}

\graphicspath{ {./figures/} }

\title{Octopus(ish) Model}
\author{Patrick C. Shriwise}
\institute{University of Wisconsin - Madison}


%--- Title Frame ------------%
\maketitle


%---- Slide 1 ---------------%
\begin{frame}
\frametitle{Concept}
\begin{center}
Origin of the idea for an octopus model:
\vfill
\includegraphics[scale=0.1]{Octopus_concept}
\end{center}

\end{frame}

%---- Slide 2 ---------------%
\begin{frame}
\frametitle{Construction}

\begin{columns}[T]
\begin{column}{0.48\textwidth}
Concept for arms:
\begin{itemize}
\item B-spline curve path to define arm
\item pre-defined radial profile
\item sample path and generate circular cross-setions
\end{itemize}
\end{column}

\hfill

\begin{column}{0.48\textwidth}
\begin{center}
\includegraphics[scale=0.32, trim = 100 0 0 0]{radial_profile}
\end{center}
\end{column}
\end{columns}

\end{frame}


%---- Slide 3 ---------------%
\begin{frame}
\frametitle{Construction}

\begin{center}
\includegraphics[scale=0.2, trim = 175 0 0 0]{arm_ctrl_cage}
\end{center}

\end{frame}


%---- Slide 4 ---------------%
\begin{frame}
\frametitle{Construction}

\begin{center}
\includegraphics[scale=0.4]{arm_n_shoulder}\\
Body connection (shoulder included) 

\end{center}

\end{frame}

%---- Slide 5 ---------------%
\begin{frame}
\frametitle{Problems}

\begin{columns}
%Column 1
\begin{column}{0.48\textwidth}
Issues with this method:
\begin{itemize}
\item surface torsion caused 'kinks' in the arm
\item comes from calculation of normal vector for arm's control cage
\end{itemize}
\end{column}

\hfill

%Column 2
\begin{column}{0.48\textwidth}

\begin{center}
\includegraphics[scale=0.25, clip = true, trim= 100 100 200 100]{torsion_ex}
\end{center}

\end{column}


\end{columns}

\end{frame}

%---- Slide 5 ---------------%
\begin{frame}
\frametitle{Simplification and Body}


\begin{columns}
%Column 1
\begin{column}{0.48\textwidth}
\begin{center}
\includegraphics[scale=0.15]{octo_no_body}
\end{center}
\end{column}

\hfill

%Column 2
\begin{column}{0.48\textwidth}

\begin{center}
\includegraphics[scale=0.33, trim = 100 0 0 0 ]{partial_body}
\end{center}

\end{column}


\end{columns}

\end{frame}

%---- Slide 6 ---------------%
\begin{frame}
\frametitle{Closing the Model}


\begin{center}
\includegraphics[scale=0.15]{octo_no_armpits}
\end{center}

\vfill


\begin{center}
\includegraphics[scale=0.15, clip=true, trim = 200 200 230 200]{shoulders_n_armpits}
\end{center}


\end{frame}

%---- Slide 7 ---------------%
\begin{frame}
\frametitle{Final Model}

\begin{center}
After fixing some surface senses...
\includegraphics[scale=0.25]{octo} \\
(MeshLab rendering)
\end{center}

\end{frame}

\end{document}


